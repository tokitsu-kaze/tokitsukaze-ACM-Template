\documentclass[a4paper,twocolumn]{article}
\usepackage[UTF8]{ctex}
\usepackage{setspace}
\usepackage{amsmath, amsthm}
\usepackage{listings,xcolor}
\usepackage{geometry} % 设置页边距
\usepackage{fontspec}
\usepackage{graphicx}
\usepackage{fancyhdr} % 自定义页眉页脚
\setsansfont{Consolas} % 设置英文字体
\setmonofont[Mapping={}]{Consolas} % 英文引号之类的正常显示,相当于设置英文字体
\geometry{left=1cm,right=1cm,top=2cm,bottom=0.5cm} % 页边距
\setlength{\columnsep}{30pt}
% \setlength\columnseprule{0.4pt} % 分割线
%==============================常用宏包、环境==============================%

%==============================页眉、页脚、代码格式设置==============================%
% 页眉、页脚设置
\pagestyle{fancy}
% \lhead{CUMTB}
\lhead{\CJKfamily{hei} ACM-ICPC Template by tokitsukaze}
\chead{}
% \rhead{Page \thepage}
\rhead{\CJKfamily{hei} 第 \thepage 页}
\lfoot{}
\cfoot{}
\rfoot{}
\renewcommand{\headrulewidth}{0.4pt}
\renewcommand{\footrulewidth}{0.4pt}




\lstset{
    language    = c++,
    numbers     = left,
    numberstyle = \tiny,
    breaklines  = true,
    captionpos  = b,
    tabsize     = 4,
    frame       = single,
    columns     = fullflexible,
    commentstyle = \color[RGB]{0,128,0},
    keywordstyle = \color[RGB]{0,0,255},
    basicstyle   = \small\ttfamily,
    stringstyle  = \color[RGB]{148,0,209}\ttfamily,
    rulesepcolor = \color{red!20!green!20!blue!20},
    showstringspaces = false,
}


\title{ACM-ICPC Template}
\author {tokitsukaze}

\begin{document}
\begin{titlepage}
\maketitle
\vspace*{50pt}
\begin{center}

\includegraphics[width=4in]{1.png}
\end{center}
\end{titlepage}
\newpage
\pagestyle{empty}
\renewcommand{\contentsname}{目录}
\tableofcontents
\newpage\clearpage
\newpage
\pagestyle{fancy}
\setcounter{page}{1}

\section{字符串}
\subsection{KMP}
\subsubsection{KMP}
\lstinputlisting{字符串/KMP/KMP.cpp}
\subsubsection{exKMP}
\lstinputlisting{字符串/KMP/exKMP.cpp}
\subsection{hash}
\subsubsection{hash}
\lstinputlisting{字符串/hash/hash.cpp}
\subsubsection{good\_hash\_prime}
\lstinputlisting{字符串/hash/good_hash_prime.cpp}
\subsubsection{hash\_map}
\lstinputlisting{字符串/hash/hash_map.cpp}
\subsubsection{BKDRHash}
\lstinputlisting{字符串/hash/BKDRHash.cpp}
\subsection{Manacher}
\subsubsection{插字符}
\lstinputlisting{字符串/Manacher/Manacher-插字符.cpp}
\subsubsection{不插字符}
\lstinputlisting{字符串/Manacher/Manacher-不插字符.cpp}
\subsection{后缀数组}
\subsubsection{倍增sa}
\lstinputlisting{字符串/后缀数组/倍增sa.cpp}
\subsubsection{SA-IS}
\lstinputlisting{字符串/后缀数组/sais.cpp}
\subsection{自动机}
\subsubsection{AC自动机}
\lstinputlisting{字符串/自动机/AC自动机.cpp}
\subsubsection{大字符集AC自动机}
\lstinputlisting{字符串/自动机/大字符集AC自动机.cpp}
\subsubsection{回文自动机}
\lstinputlisting{字符串/自动机/回文自动机.cpp}
\subsubsection{序列自动机}
\lstinputlisting{字符串/自动机/序列自动机.cpp}
\subsubsection{KMP自动机}
\lstinputlisting{字符串/自动机/KMP自动机.cpp}
\subsubsection{后缀自动机}
\lstinputlisting{字符串/自动机/后缀自动机.cpp}
\subsubsection{广义后缀自动机}
\lstinputlisting{字符串/自动机/广义后缀自动机.cpp}
\subsection{最小表示法}
\subsubsection{最小表示法}
\lstinputlisting{字符串/最小表示法/最小表示法.cpp}
\subsubsection{最大表示法}
\lstinputlisting{字符串/最小表示法/最大表示法.cpp}
\subsection{shift\_and}
\lstinputlisting{字符串/shift_and.cpp}
\section{数据结构}
\subsection{离散化}
\lstinputlisting{数据结构/离散化.cpp}
\subsection{RMQ}
\subsubsection{一维RMQ}
\lstinputlisting{数据结构/RMQ/一维RMQ.cpp}
\subsubsection{二维RMQ}
\lstinputlisting{数据结构/RMQ/二维RMQ.cpp}
\subsection{单调队列}
\lstinputlisting{数据结构/单调队列.cpp}
\subsection{并查集}
\subsubsection{并查集}
\lstinputlisting{数据结构/并查集/并查集.cpp}
\subsubsection{map实现并查集}
\lstinputlisting{数据结构/并查集/map实现并查集.cpp}
\subsubsection{可撤销并查集}
\lstinputlisting{数据结构/并查集/可撤销并查集.cpp}
\subsection{树状数组}
\subsubsection{一维单点BIT}
\lstinputlisting{数据结构/树状数组/一维单点BIT.cpp}
\subsubsection{一维区间BIT}
\lstinputlisting{数据结构/树状数组/一维区间BIT.cpp}
\subsubsection{二维单点BIT}
\lstinputlisting{数据结构/树状数组/二维单点BIT.cpp}
\subsection{线段树}
\subsubsection{线段树}
\lstinputlisting{数据结构/线段树/线段树.cpp}
\subsubsection{动态开点线段树}
\lstinputlisting{数据结构/线段树/动态开点线段树.cpp}
\subsubsection{区间查询最大子段和}
\lstinputlisting{数据结构/线段树/区间查询最大子段和.cpp}
\subsubsection{矩形面积并}
\lstinputlisting{数据结构/线段树/矩形面积并.cpp}
\subsubsection{线段树维护hash}
\lstinputlisting{数据结构/线段树/线段树维护hash.cpp}
\subsection{平衡树}
\subsubsection{Treap}
\lstinputlisting{数据结构/平衡树/Treap.cpp}
\subsubsection{Splay维护序列}
\lstinputlisting{数据结构/平衡树/Splay维护序列.cpp}
\subsubsection{FHQ-Treap维护序列}
\lstinputlisting{数据结构/平衡树/FHQ-Treap维护序列.cpp}
\subsubsection{pbds}
\lstinputlisting{数据结构/平衡树/pbds.cpp}
\subsection{字典树}
\subsubsection{trie}
\lstinputlisting{数据结构/字典树/trie.cpp}
\subsubsection{01trie}
\lstinputlisting{数据结构/字典树/01trie.cpp}
\subsection{可持久化}
\subsubsection{可持久化线段树}
\lstinputlisting{数据结构/可持久化/可持久化线段树.cpp}
\subsubsection{可持久化01trie}
\lstinputlisting{数据结构/可持久化/可持久化01trie.cpp}
\subsubsection{可持久化数组}
\lstinputlisting{数据结构/可持久化/可持久化数组.cpp}
\subsubsection{可持久化并查集}
\lstinputlisting{数据结构/可持久化/可持久化并查集.cpp}
\subsection{树套树}
\subsubsection{线段树套线段树}
\lstinputlisting{数据结构/树套树/线段树套线段树.cpp}
\subsubsection{线段树套treap}
\lstinputlisting{数据结构/树套树/线段树套treap.cpp}
\subsubsection{树状数组套treap}
\lstinputlisting{数据结构/树套树/树状数组套treap.cpp}
\subsection{李超树}
\lstinputlisting{数据结构/李超树.cpp}
\subsection{kd-tree}
\lstinputlisting{数据结构/kd-tree.cpp}
\subsection{可并堆}
\subsubsection{pbds可并堆}
\lstinputlisting{数据结构/可并堆/pbds可并堆.cpp}
\subsubsection{左偏树(支持打tag,不支持删除)}
\lstinputlisting{数据结构/可并堆/左偏树支持打tag,不支持删除.cpp}
\subsubsection{左偏树(支持删除,不支持打tag)}
\lstinputlisting{数据结构/可并堆/左偏树支持删除,不支持打tag.cpp}
\subsection{k叉哈夫曼树}
\begin{spacing}{1.5}
\input{数据结构/k叉哈夫曼树.tex}
\end{spacing}
\lstinputlisting{数据结构/k叉哈夫曼树.cpp}
\subsection{笛卡尔树}
\begin{spacing}{1.5}
\input{数据结构/笛卡尔树.tex}
\end{spacing}
\lstinputlisting{数据结构/笛卡尔树.cpp}
\subsection{析合树}
\begin{spacing}{1.5}
\input{数据结构/析合树.tex}
\end{spacing}
\lstinputlisting{数据结构/析合树.cpp}
\subsection{莫队算法}
\lstinputlisting{数据结构/莫队算法.cpp}
\subsection{ODT}
\lstinputlisting{数据结构/ODT.cpp}
\subsection{分块}
\lstinputlisting{数据结构/Block.cpp}
\section{树}
\subsection{LCA}
\subsubsection{倍增LCA}
\lstinputlisting{树/LCA/倍增LCA轻便版.cpp}
\subsubsection{RMQ维护欧拉序求LCA}
\lstinputlisting{树/LCA/RMQ维护欧拉序求LCA.cpp}
\subsection{树链剖分}
\subsubsection{轻重链剖分}
\lstinputlisting{树/树链剖分/轻重链剖分.cpp}
\subsection{树的重心}
\lstinputlisting{树/树的重心.cpp}
\subsection{树上倍增}
\lstinputlisting{树/树上倍增.cpp}
\subsection{树hash}
\lstinputlisting{树/树hash.cpp}
\subsection{虚树}
\lstinputlisting{树/虚树.cpp}
\subsection{树分治}
\subsubsection{点分治}
\lstinputlisting{树/树分治/点分治.cpp}
\subsection{Link-Cut-Tree}
\lstinputlisting{树/Link-Cut-Tree.cpp}
\section{图论}
\subsection{链式前向星}
\lstinputlisting{图论/链式前向星.cpp}
\subsection{最短路}
\subsubsection{dijkstra}
\lstinputlisting{图论/最短路/dijkstra.cpp}
\subsubsection{spfa}
\lstinputlisting{图论/最短路/spfa.cpp}
\subsubsection{floyd求最小环}
\lstinputlisting{图论/最短路/floyd求最小环.cpp}
\subsubsection{Johnson}
\lstinputlisting{图论/最短路/Johnson.cpp}
\subsubsection{同余最短路}
\lstinputlisting{图论/最短路/同余最短路.cpp}
\subsection{最小生成树}
\subsubsection{kruskal}
\lstinputlisting{图论/最小生成树/kruskal.cpp}
\subsubsection{kruskal重构树}
\lstinputlisting{图论/最小生成树/kruskal重构树.cpp}
\subsubsection{prim}
\lstinputlisting{图论/最小生成树/prim.cpp}
\subsection{二分图匹配}
\subsubsection{二分图最大匹配}
\lstinputlisting{图论/二分图匹配/二分图最大匹配.cpp}
\subsubsection{二分图最大权完美匹配}
\lstinputlisting{图论/二分图匹配/二分图最大权完美匹配.cpp}
\subsection{最大流}
\subsubsection{dinic}
\lstinputlisting{图论/最大流/dinic.cpp}
\subsubsection{有源汇上下界网络流}
\lstinputlisting{图论/最大流/有源汇上下界网络流.cpp}
\subsection{费用流}
\subsubsection{spfa费用流}
\lstinputlisting{图论/费用流/spfa费用流.cpp}
\subsubsection{dijkstra费用流(dij求h)}
\lstinputlisting{图论/费用流/dijkstra费用流-dij求h.cpp}
\subsubsection{dijkstra费用流(spfa求h)}
\lstinputlisting{图论/费用流/dijkstra费用流-spfa求h.cpp}
\subsection{连通性}
\subsubsection{强连通分量}
\lstinputlisting{图论/连通性/强连通分量.cpp}
\subsubsection{边双连通分量}
\lstinputlisting{图论/连通性/边双连通分量.cpp}
\subsubsection{点双连通分量}
\lstinputlisting{图论/连通性/点双连通分量.cpp}
\subsubsection{圆方树}
\lstinputlisting{图论/连通性/圆方树.cpp}
\subsection{团}
\subsubsection{最大团}
\lstinputlisting{图论/团/最大团.cpp}
\subsubsection{极大团计数}
\lstinputlisting{图论/团/极大团计数.cpp}
\subsection{拓扑排序}
\lstinputlisting{图论/拓扑排序.cpp}
\subsection{2-sat}
\subsubsection{2-sat输出任意解}
\lstinputlisting{图论/2-sat/2-sat输出任意解.cpp}
\subsubsection{2-sat字典序最小解}
\lstinputlisting{图论/2-sat/2-sat字典序最小解.cpp}
\subsection{支配树}
\lstinputlisting{图论/支配树.cpp}
\subsection{最小斯坦纳树}
\lstinputlisting{图论/最小斯坦纳树.cpp}
\section{数论}
\subsection{素数筛}
\subsubsection{埃筛}
\lstinputlisting{数论/素数筛/埃筛.cpp}
\subsubsection{线性筛}
\lstinputlisting{数论/素数筛/线性筛.cpp}
\subsubsection{区间筛}
\lstinputlisting{数论/素数筛/区间筛.cpp}
\subsection{逆元}
\subsubsection{exgcd求逆元}
\lstinputlisting{数论/逆元/exgcd求逆元.cpp}
\subsubsection{线性预处理}
\lstinputlisting{数论/逆元/线性预处理.cpp}
\subsection{扩展欧几里得}
\subsubsection{exgcd}
\lstinputlisting{数论/扩展欧几里得/exgcd.cpp}
\subsubsection{ax+by=c}
\lstinputlisting{数论/扩展欧几里得/ax+by=c.cpp}
\subsection{中国剩余定理}
\subsubsection{CRT}
\lstinputlisting{数论/中国剩余定理/CRT.cpp}
\subsubsection{exCRT}
\lstinputlisting{数论/中国剩余定理/exCRT.cpp}
\subsection{欧拉函数}
\begin{spacing}{1.5}
\input{数论/欧拉函数.tex}
\end{spacing}
\subsubsection{直接求}
\lstinputlisting{数论/欧拉函数/直接求.cpp}
\subsubsection{线性筛}
\lstinputlisting{数论/欧拉函数/线性筛.cpp}
\subsection{莫比乌斯函数}
\lstinputlisting{数论/莫比乌斯函数.cpp}
\subsection{Berlekamp-Massey}
\lstinputlisting{数论/Berlekamp-Massey.cpp}
\subsection{exBSGS}
\lstinputlisting{数论/exBSGS.cpp}
\subsection{Miller\_Rabin+Pollard\_rho}
\lstinputlisting{数论/Miller_Rabin+Pollard_rho.cpp}
\subsection{第二类Stirling数}
\lstinputlisting{数论/第二类Stirling数.cpp}
\subsection{原根}
\begin{spacing}{1.5}
\input{数论/原根.tex}
\end{spacing}
\lstinputlisting{数论/原根.cpp}
\subsection{二次剩余}
\lstinputlisting{数论/二次剩余.cpp}
\section{组合数学}
\subsection{组合数}
\subsubsection{公式预处理}
\lstinputlisting{组合数学/组合数/公式预处理.cpp}
\subsubsection{杨辉三角递推}
\lstinputlisting{组合数学/组合数/杨辉三角递推.cpp}
\subsection{卢卡斯定理}
\subsubsection{Lucas}
\lstinputlisting{组合数学/卢卡斯定理/Lucas.cpp}
\subsubsection{exLucas}
\lstinputlisting{组合数学/卢卡斯定理/exLucas.cpp}
\subsection{卡特兰数公式}
\lstinputlisting{组合数学/卡特兰数公式.cpp}
\section{多项式}
\subsection{FFT}
\lstinputlisting{多项式/FFT.cpp}
\subsection{NTT}
\lstinputlisting{多项式/NTT.cpp}
\subsection{FWT}
\lstinputlisting{多项式/FWT.cpp}
\subsection{拉格朗日插值}
\lstinputlisting{多项式/拉格朗日插值.cpp}
\section{矩阵}
\subsection{矩阵类}
\lstinputlisting{矩阵/矩阵类.cpp}
\subsection{高斯消元}
\subsubsection{浮点数方程组}
\lstinputlisting{矩阵/高斯消元/浮点数方程组.cpp}
\subsubsection{异或方程组}
\lstinputlisting{矩阵/高斯消元/异或方程组.cpp}
\subsubsection{同余方程组}
\lstinputlisting{矩阵/高斯消元/同余方程组.cpp}
\subsection{线性基}
\subsubsection{线性基}
\lstinputlisting{矩阵/线性基/线性基.cpp}
\subsubsection{带删除线性基}
\lstinputlisting{矩阵/线性基/带删除线性基.cpp}
\section{博弈}
\subsection{sg函数}
\lstinputlisting{博弈/sg函数.cpp}
\subsection{结论}
\begin{spacing}{1.5}
-----------------------------------------------------------------------\\
1.阶梯博弈\\
0层为终点的阶梯博弈,等价于奇数层的nim,偶数层的移动不影响结果 \\
\\
2.SJ定理\\
对于任意一个Anti-SG游戏,如果我们规定当局面中所有的单一游戏的SG值为0时,游戏结束。\\
先手必胜当且仅当:\\
(1)游戏的SG函数不为0且游戏中某个单一游戏的SG函数大于1;\\
(2)游戏的SG函数为0且游戏中没有单一游戏的SG函数大于1。\\
\\
3.k-nim\\
\\
4.树上删边博弈\\
\\
\end{spacing}
\section{dp}
\subsection{LIS}
\lstinputlisting{dp/LIS.cpp}
\subsection{LPS}
\lstinputlisting{dp/LPS.cpp}
\subsection{数位dp}
\lstinputlisting{dp/数位dp.cpp}
\section{杂项}
\subsection{FastIO}
\lstinputlisting{杂项/FastIO.cpp}
\subsection{O(1)快速乘}
\lstinputlisting{杂项/快速乘.cpp}
\subsection{快速模}
\lstinputlisting{杂项/快速模.cpp}
\subsection{xor\_sum(1,n)}
\lstinputlisting{杂项/xor_sum.cpp}
\subsection{约瑟夫环kth}
\lstinputlisting{杂项/约瑟夫环kth.cpp}
\subsection{判断星期几}
\lstinputlisting{杂项/判断星期几.cpp}
\subsection{整数三分}
\lstinputlisting{杂项/整数三分.cpp}
\subsection{有根树与prufer序列的转换}
\lstinputlisting{杂项/tree_and_prufer.cpp}
\subsection{网格整数点正方形个数}
\lstinputlisting{杂项/网格整数点正方形个数.cpp}
\subsection{模拟退火}
\lstinputlisting{杂项/模拟退火.cpp}
\subsection{斐波那契01串的第k个字符}
\lstinputlisting{杂项/kth_FibString.cpp}
\subsection{光速幂}
\lstinputlisting{杂项/光速幂.cpp}
\subsection{倍增求等比数列和}
\lstinputlisting{杂项/倍增求等比数列和.cpp}
\subsection{矩阵旋转90度}
\lstinputlisting{杂项/矩阵旋转90度.cpp}
\section{附录}
\subsection{NTT常用模数}
\begin{spacing}{1.5}
\input{附录/NTT常用模数.tex}
\end{spacing}
\subsection{线性基求交}
\lstinputlisting{附录/线性基求交.cpp}
\end{document}